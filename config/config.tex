% Configuración de codificación y fuentes
\usepackage[utf8]{inputenc}      % Permite usar caracteres UTF-8 en el archivo fuente.
\usepackage[T1]{fontenc}         % Usa codificación de fuente T1 para caracteres extendidos.
\usepackage{lmodern}             % Fuente moderna de LaTeX que mejora la visualización en PDF.
\usepackage[spanish]{babel}      % Configuración para español (división silábica y traducción de elementos como "Capítulo").

% Paquetes matemáticos
\usepackage{amsmath, amsthm, amssymb, amsfonts, calrsfs} % Paquetes para símbolos matemáticos, teoremas, y fuentes adicionales.

% Configuración de márgenes
\usepackage[a4paper, margin=2.5cm]{geometry} % Define los márgenes del documento (2.5 cm en este caso).

% Encabezados y pies de página
\usepackage{fancyhdr}            % Permite personalizar encabezados y pies de página.

% Enlaces y referencias
\usepackage[hidelinks]{hyperref} % Habilita enlaces en el PDF ocultando los colores.

% Configuración de espaciado
\usepackage{setspace}            % Permite ajustar el espaciado entre líneas.
\onehalfspacing                  % Establece espaciado de 1.5 líneas en todo el documento.

% Configuración de encabezados
\setlength{\headheight}{15.5pt}                                 % Ajusta la altura del encabezado para evitar advertencias.
\pagestyle{fancy}                                               % Activa el estilo de página personalizado con "fancyhdr".
\fancyhead{}                                                    % Limpia los encabezados predeterminados.
\renewcommand{\sectionmark}[1]{\markboth{\MakeUppercase{#1}}{}} % Redefinir el contenido de \leftmark para que solo incluya el título de la sección
\fancyhead[C]{\leftmark}                                        % Muestra el nombre del capítulo centrado en el encabezado.
\fancyfoot{}                                                    % Limpia los pies de página predeterminados.
\fancyfoot[C]{\thepage}                                         % Muestra el número de página centrado en el pie de página.

% Definición de entornos para teoremas y otros conceptos matemáticos
\newtheorem{theorem}{Teorema}[section]       % Define el entorno "Teorema", numerado por capítulo.
\newtheorem{prop}{Proposición}[section]      % Define "Proposición", numerado por capítulo.
\newtheorem{lem}{Lema}[section]              % Define "Lema", numerado por capítulo.
\newtheorem{cor}{Corolario}[section]         % Define "Corolario", numerado por capítulo.
\theoremstyle{definition}                    % Cambia el estilo para definiciones y ejemplos.
\newtheorem{definition}{Definición}[section] % Define "Definición", numerado por capítulo.
\newtheorem{note}{Nota}[section]             % Define "Definición", numerado por capítulo.
\newtheorem{example}{Ejemplo}[section]       % Define "Ejemplo", numerado por capítulo.
\newtheorem{obs}{Observación}[section]       % Define "Observación", numerado por capítulo.

% Comandos para personalizar notación matemática
\renewcommand*{\deg}{\normalfont\text{gr}\hspace*{1mm}} % Redefine el símbolo de grados como "gr".
\renewcommand{\vec}[1]{\mathbf{#1}}                     % Define vectores como negritas.
\newcommand{\cl}[1]{\overline{#1}}                      % Define la clausura.
\newcommand{\conj}[1]{\overline{#1}}                    % Define conjugación.
\newcommand{\fr}[1]{\partial{#1}}                       % Define frontera.
\newcommand{\ip}[2]{\left\langle #1,#2\right\rangle}    % Define producto interno.
\newcommand{\norm}[1]{\left\| #1\right\|}               % Define norma.
\newcommand{\ec}[1]{\left[#1\right]}                    % Define corchetes.
\newcommand{\set}[1]{\left\lbrace#1\right\rbrace}       % Define conjuntos.
\newcommand{\suc}[1]{\left\lbrace#1\right\rbrace}       % Define sucesiones.
\newcommand{\abs}[1]{\left| #1\right|}                  % Define valor absoluto.
\newcommand{\ent}[1]{\left\lfloor #1\right\rfloor}      % Define parte entera.
\newcommand{\eps}{\varepsilon}                          % Redefine epsilon.
\renewcommand{\Im}{\operatorname{Im}}                   % Redefinición del operador "Parte imaginaria"
\renewcommand{\Re}{\operatorname{Re}}                   % Redefinición del operador "Parte real"
\DeclareMathOperator{\Log}{Log}                         % Operador "Logaritmo"
\DeclareMathOperator{\inter}{int}                       % Operador "Interior"
\DeclareMathOperator{\mcm}{mcm}                         % Operador "mínimo común múltiplo"
\DeclareMathOperator{\mcd}{mcd}                         % Operador "máximo común divisor"
\DeclareMathOperator{\Cov}{Cov}                         % Operador matemático "Covarianza"

% Notación para conjuntos numéricos
\def\NN{\mathbb{N}} % Números naturales
\def\ZZ{\mathbb{Z}} % Números enteros
\def\QQ{\mathbb{Q}} % Números racionales
\def\RR{\mathbb{R}} % Números reales
\def\CC{\mathbb{C}} % Números complejos

% Notación para vectores
\def\0{\vec{0}}
\def\1{\vec{1}}
\def\a{\vec{a}}
\def\b{\vec{b}}
\def\u{\vec{u}}
\def\p{\vec{p}}
\def\t{\vec{t}}
\def\x{\vec{x}}
\def\y{\vec{y}}
\def\z{\vec{z}}
\def\f{\vec{f}}
\def\g{\vec{g}}


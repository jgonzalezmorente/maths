\documentclass[a4paper,12pt]{article}

% Codificación y fuentes
\usepackage[utf8]{inputenc}
\usepackage[T1]{fontenc}
\usepackage{lmodern}
\usepackage[spanish]{babel}

% Paquetes matemáticos
\usepackage{amsmath, amsthm, amssymb, amsfonts}

% Configuración de enlaces
\usepackage[hidelinks]{hyperref}

% Márgenes de página
\usepackage[a4paper, margin=2.5cm]{geometry}

% Configuraciones de teoremas y entornos personalizados
\newtheorem{theorem}{Teorema}[section]
\newtheorem{prop}{Proposición}[section]
\newtheorem{lem}{Lema}[section]
\newtheorem{cor}{Corolario}[section]
\theoremstyle{definition}
\newtheorem{df}{Definición}[section]
\newtheorem{example}{Ejemplo}[section]
\newtheorem{obs}{Observación}[section]

% Comandos personalizados
\renewcommand*{\deg}{\normalfont\text{gr}\hspace*{1mm}}
\renewcommand{\Im}{\operatorname{\rm Im}}
\renewcommand{\Re}{\operatorname{\rm Re}}
\renewcommand{\vec}[1]{\mathbf{#1}}
\newcommand{\Log}{\operatorname{\rm Log}}
\newcommand{\inter}{\operatorname{\rm int}}
\newcommand{\cl}[1]{\overline{#1}}
\newcommand{\conj}[1]{\overline{#1}}
\newcommand{\fr}[1]{\partial{#1}}
\newcommand{\ip}[2]{\left\langle #1,#2\right\rangle}
\newcommand{\norm}[1]{\left\| #1\right\|}
\newcommand{\mcm}[1]{\operatorname{\rm mcm}\left(#1\right)}
\newcommand{\mcd}[1]{\operatorname{\rm mcd}\left(#1\right)}
\newcommand{\ec}[1]{\left[#1\right]}
\newcommand{\set}[1]{\left\lbrace#1\right\rbrace}
\newcommand{\suc}[1]{\left\lbrace#1\right\rbrace}
\newcommand{\abs}[1]{\left| #1\right|}
\newcommand{\ent}[1]{\left\lfloor #1\right\rfloor}
\newcommand{\eps}{\varepsilon}

% Definición de conjuntos
\def\NN{\mathbb{N}}
\def\ZZ{\mathbb{Z}}
\def\QQ{\mathbb{Q}}
\def\RR{\mathbb{R}}
\def\CC{\mathbb{C}}

% Definición de vectores
\def\0{\vec{0}}
\def\a{\vec{a}}
\def\b{\vec{b}}
\def\u{\vec{u}}
\def\p{\vec{p}}
\def\t{\vec{t}}
\def\x{\vec{x}}
\def\y{\vec{y}}
\def\z{\vec{z}}
\def\f{\vec{f}}
\def\g{\vec{g}}
\def\p{\vec{p}}
\def\t{\vec{t}}


\title{Teorema de Bolzano-Weierstrass}
\author{José Antonio González Morente}

\begin{document}
\maketitle

\begin{abstract}
    En este documento se presenta el \textbf{Teorema de Bolzano-Weierstrass} en $\RR^n$, el cual afirma que todo subconjunto infinito y acotado de $\RR^n$ tiene al menos un punto de acumulación. Se ofrece una demostración detallada y mejorada, enfatizando la claridad en la notación y la rigurosidad en los argumentos. La demostración utiliza una construcción de intervalos $n$-dimensionales anidados cuya longitud disminuye exponencialmente, asegurando la convergencia hacia un punto límite que es un punto de acumulación del conjunto dado.
\end{abstract}

\begin{thm}[\textbf{Bolzano-Weierstrass}]
Si un conjunto acotado $S$ de $\RR^n$ contiene una infinidad de puntos, entonces existe al menos un punto $\t \in \RR^n$ que es punto de acumulación de $S$.
\end{thm}

\begin{proof}
Como $S$ es acotado, existe un número real positivo $a > 0$ tal que $S \subset B(\0, a)$, donde $B(\0, a)$ es la bola $n$-dimensional centrada en el origen con radio $a$. Por lo tanto, $S$ está contenido en el intervalo $n$-dimensional:
$$J_1 = I_1^{(1)} \times I_2^{(1)} \times \dots \times I_n^{(1)},$$
donde cada $I_k^{(1)} = [-a, a]$ para $k = 1, 2, \dots, n$.

\begin{itemize}
    \item \textbf{Paso 1: Subdivisión inicial}.

    Dividimos cada intervalo $I_k^{(1)}$ en dos subintervalos iguales:
    $$I_{k,1}^{(1)} = \left[-a, 0\right], \quad I_{k,2}^{(1)} = \left[0, a\right]$$
    Consideramos todos los productos cartesianos de la forma:
    $$J = I_{1, k_1}^{(1)} \times I_{2, k_2}^{(1)} \times \dots \times I_{n, k_n}^{(1)}$$
    donde cada $k_i \in \{1, 2\}$. Hay exactamente $2^n$ intervalos $n$-dimensionales de este tipo, y su unión es $J_1$. Dado que $S$ es infinito y está contenido en $J_1$, al menos uno de estos intervalos, denotado por $J_2$, contiene infinitos puntos de $S$.

    \item \textbf{Paso 2: Proceso iterativo}.

    Repetimos el proceso con $J_2$:
    \begin{enumerate}
        \item Dividimos cada intervalo $I_k^{(2)}$ en dos subintervalos iguales:
        $$I_{k,1}^{(2)} \text{ y } I_{k,2}^{(2)}$$
        donde $I_k^{(2)}$ es el intervalo correspondiente en $J_2$.
        \item Obtenemos $2^n$ nuevos intervalos $n$-dimensionales cuya unión es $J_2$.
        \item Al menos uno de estos intervalos, denotado por $J_3$, contiene infinitos puntos de $S$.
    \end{enumerate}

    \item \textbf{Paso 3: Construcción de la sucesión de intervalos anidados}.

    Continuamos este procedimiento inductivamente, obteniendo una sucesión de intervalos $n$-dimensionales anidados:
    $$J_1 \supset J_2 \supset J_3 \supset \dots$$
    donde cada $J_m$ se puede expresar como:
    $$J_m = I_1^{(m)} \times I_2^{(m)} \times \dots \times I_n^{(m)}$$
    con $I_k^{(m)} \subset I_k^{(m-1)}$ y longitud:
    $$\ell_k^{(m)} = b_k^{(m)} - a_k^{(m)} = \frac{a}{2^{m-2}}$$
    para $k = 1, 2, \dots, n$ y $m \geq 1$.

    \item \textbf{Paso 4: Determinación del punto límite}.

    Para cada $k$, las sucesiones $\{a_k^{(m)}\}_{m=1}^\infty$ y $\{b_k^{(m)}\}_{m=1}^\infty$ cumplen:
    \begin{enumerate}
        \item $\{a_k^{(m)}\}$ es creciente y acotada superiormente, por lo que converge al supremo $t_k = \lim_{m \to \infty} a_k^{(m)}$.
        \item $\{b_k^{(m)}\}$ es decreciente y acotada inferiormente, por lo que converge al ínfimo $t_k' = \lim_{m \to \infty} b_k^{(m)}$.
        \item Como $b_k^{(m)} - a_k^{(m)} = \frac{a}{2^{m-2}} \to 0$ cuando $m \to \infty$, se sigue que $t_k = t_k'$.
    \end{enumerate}
    Por lo tanto, ambas sucesiones convergen al mismo límite $t_k$, y podemos afirmar que:
    $$\lim_{m \to \infty} a_k^{(m)} = \lim_{m \to \infty} b_k^{(m)} = t_k$$

    Definimos el punto $\t = (t_1, t_2, \dots, t_n) \in \RR^n$. Este punto pertenece a todos los intervalos $J_m$:
    $$\t \in \bigcap_{m=1}^\infty J_m$$

    \item \textbf{Paso 5: Demostración de que $\t$ es punto de acumulación de $S$}.

    Sea $\varepsilon > 0$ arbitrario. Elegimos $m$ suficientemente grande tal que:
    $$\frac{a}{2^{m-2}} < \frac{\varepsilon}{\sqrt{n}}$$
    Entonces, el diámetro de $J_m$ satisface:
    $$\text{diam}(J_m) = \sqrt{\sum_{k=1}^n \left(\ell_k^{(m)}\right)^2} \leq \sqrt{n \left(\frac{a}{2^{m-2}}\right)^2} < \varepsilon$$
    Por lo tanto,
    $$J_m \subset B(\t, \varepsilon)$$
    Como $J_m$ contiene infinitos puntos de $S$, se deduce que $B(\t, \varepsilon)$ también contiene infinitos puntos de $S$. Dado que $\varepsilon > 0$ es arbitrario, concluimos que $\t$ es un punto de acumulación de $S$.
\end{itemize}
\end{proof}
\begin{thebibliography}{9}
    \bibitem{apostol}
    Tom M. Apostol, \textit{Análisis Matemático}, Reverté, 2da Edición, 1976.
\end{thebibliography}
\end{document}
